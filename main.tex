\documentclass[a4paper,11pt,AutoFakeBold]{ctexart}
\usepackage{array}
\usepackage{amsmath}
\usepackage{amssymb}
\usepackage{array}
% 关于中文文献引用的格式参数设置请参考
% https://github.com/hushidong/biblatex-gb7714-2015
\usepackage[backend=biber,style=gb7714-2015]{biblatex}
\usepackage{fancyhdr}
\usepackage[margin=1in]{geometry}
\usepackage{graphicx}
\usepackage[hidelinks]{hyperref}
\usepackage{listings}
\usepackage{minted}
\usepackage{tabularx}
\usepackage{url}
\usepackage[dvipsnames]{xcolor}

% 设置页眉,从第二页开始
\pagestyle{fancy}
\fancyhead[L]{学生姓名}
\fancyhead[C]{项目名称}
\fancyhead[R]{课程名称}
\fancyfoot[C]{\thepage}
\renewcommand{\headrulewidth}{1pt}

% 定义行距=1.25倍
\linespread{1.25}

% 定义英文字体
\setmainfont{Times New Roman}
% 定义中文字体
\setCJKmainfont{FandolSong}
% 定义生僻字处理,当文字无法显示时前缀指令`\fallback`
\setCJKfamilyfont{Babel}{BabelStone Han}
\newcommand{\fallback}{\CJKfamily{Babel}}

% 设置一级标题左对齐
\ctexset{
  section={
    format+ =\raggedright
  }
}

% 定义常见软链颜色
\hypersetup{
  colorlinks = true,
  urlcolor = CadetBlue,
  linkcolor = Cerulean,
  citecolor = Maroon
}

% 定义行内代码格式
\definecolor{light-gray}{rgb}{0.96, 0.96, 0.96}
\NewDocumentCommand{
  \codeword}{v}{%
    \colorbox{light-gray}{
      \texttt{\textcolor{Black}{#1}
    }
  }%
}

% 定义引用源
\addbibresource{ref.bib}

\title{\textbf{上海交通大学-课程项目报告(非官方模版)}}
\author{白\fallback{鹡鸰}}
\date{}

\begin{document}

\maketitle

\section{关于此模版}

\subsection{适用范围}

本文档是由上海交通大学控制导论课的助教搭建的专题写作项目模板,遵循MIT协议开源,欢迎其他课程的同学使用。

\subsection{基本参数}

此模版基于 \codeword{ctexart} 模版稍作调整,并增加了一些常用包和相应设置。目前仅在Overleaf上使用 \textsc{XeLaX} 编译并通过了测试。

\subsubsection{字体}

\begin{itemize}
    \item 正文采用了了11号字体。
    \item 中文正文默认字体为 \codeword{FandolSong},英文正文默认字体为 \codeword{Times New Roman}。
\end{itemize}

\subsubsection{页面设置}

\begin{itemize}
    \item 默认页边距为四边各2.54厘米。
    \item 默认行距为1.25倍。
    
\end{itemize}

\subsection{引用规范}

此模版采用了中文参考文献著录/标注标准 \codeword{GB/T 7714-2015} 的 \href{https://github.com/hushidong/biblatex-gb7714-2015?tab=readme-ov-file#jumptotutorial}{biblatex实现} \parencite{gb7714},考虑到上交投稿通常为IEEE刊物,文末引用的索引规则与IEEE对齐,采用的是\textbf{顺序编码制非上标}模式,关于该格式的更多信息请查阅IEEE官方格式文档。

\section{常见格式错误}

从课程教学中,我们收集到了以下在中文学术写作中的常见错误:

(待办事项:完善格式错误的样例、解释、与正确示范)

\subsection{英文术语相关错误}

\subsubsection{大小写规则不稳定}

\subsubsection{缩写规则不稳定}

\subsection{文末引用相关错误}

\subsubsection{空格规则部分定}

\subsubsection{索引位置错误}

\subsection{标点符号相关错误}

\subsubsection{半角与全角混用}

\section{总结与倡议}

大部分的错误并非绝对意义上的“错误”,而是因为中英文格式混用导致的视觉不适。归根到底,应当归咎于:

\begin{enumerate}
    \item 理工科的学术报告中,会涉及大量缺乏对应中文的新术语,在被迫中英文参杂的写作过程中,稍有不慎就会出现排版纰漏;
    \item 中文的学术写作缺乏一份通用的、被广泛认可的格式规范。
\end{enumerate}

目前国内社区尚未提供一份通用的、适用于课程项目和毕业论文的简体中文格式协议,这对于大学课程教学和学生学习都是一项必要的资料。因此,有必要确立并推广一份标准化格式协议,以帮助本科同学们更快地上手学术写作的基本规范。

\printbibliography[heading=bibliography,title=参考文献]

\end{document}
